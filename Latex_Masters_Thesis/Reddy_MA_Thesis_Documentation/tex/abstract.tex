\chapter*{Abstract}
\label{abstract}
\addcontentsline{toc}{chapter}{\textbf{Abstract}}

This thesis delves into the dynamic landscape of \ac{mmWave} communication technology where the main focus is on conquering the challenges posed by increased path loss at \ac{mmWave} frequencies, particularly due to obstructions. Moving towards 6G, the resulting shadow regions need to be illuminated to make \ac{mmWave} networks investments feasible, where metasurface based \ac{SRE}s offer a pivotal solution by deliberately reflecting \ac{EM} waves to enhance performance. Addressing limitations of \ac{IRS}, such as complex deployment logistics and active power requirements, have prompted a shift towards passive reflectors such as \ac{HELIOS}. This thesis aims to solve the lack of channel models for \ac{HELIOS} reflectors by leveraging it as an efficient and scalable alternative, thus, offering a promising solution for future cellular networks. The thesis also highlights the potential advantages of HELIOS reflectors over \ac{IRS} in terms of peak gain and beamwidth and emphasizes the need for optimizing the performance.

The thesis then proceeds to develop an analytical model for the bistatic \ac{RCS} for the HELIOS reflectors and validate it by comparing with EM simulations. The thesis further highlights the main reflection lobes are well-represented, e.g., in regards to gain and beamwidth and also highlights the attained speed-up and the significant reduction in computing time of using the proposed model. Further focusing on implementing the analytical model in an urban scenario to predict the connectivity and improve the initial reflector geometry and position to optimize the \ac{mmWave} communication. The result shows the better coverage and stronger signals by IRSs than HELIOS reflectors due to its dynamic nature which allows it to focus on each UE position individually. Further investigation reveals that by fixing the node for IRS, an equivalent coverage to that of \ac{HELIOS} reflectors is achieved, thereby proving its applicability in sophisticated hybrid network planning and \ac{HELIOS} customization process.

\textbf{Keywords:}
\begin{itemize}
	\item Metasurface-aided mmWave communications
	\item Simulation and modeling of HELIOS reflectors
	\item Channel and reflection modeling
\end{itemize} 