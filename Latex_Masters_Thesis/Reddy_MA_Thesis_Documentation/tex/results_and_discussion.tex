\chapter{Conclusion}\label{Conclusion}
In this master thesis, we dug into the complex environment of mmWave communication and its propagation properties, thus establishing the background for our work. Our main focus is on passive reflectors in the form of HELIOS reflectors \cite{Helios}, which demonstrates a scalable, non-reconfigurable option with immediate viability.\\
In \Cref{Simulative Analyses of HELIOS Reflector Modules}, we first conducted EM simulations to identify key components of HELIOS reflection behavior. Notably, increasing the reflector size has an impact on the reflection gain and also on the reflection beamwidth. \Cref{Analytical Modeling of HELIOS Modules} then procured an analytical HELIOS reflection model for a single module, for which we identified a matching reflection main lobe behavior in terms of peak gain as well as horizontal and vertical beamwidths. Superimposing the formula and addressing potential self-shadowing effects, \Cref{Simulation and Modeling of HELIOS Reflectors} provides the final RCS model for HELIOS consisting of $M \times N$ modules. By comparing to EM simulations, we identified, e.g., \SI{0.02}{\decibel} and \SI{0.20}{\decibel} differences for the peak gain for $1 \times 8$ and $8 \times 8$ HELIOS configurations, respectively, thus validating our analytical model. Moreover, we performed case studies to investigate the behavior of a fixed geometry along different frequencies, for varying reflecting surface sizes, and in what regard overprediction against simulation may occur. For the latter, we show that mean, median, and maximum value are rather underpredicted, which is a desirable characteristic in the context of network planning.\\
\Cref{Computing Time Analysis} then followed up with a computing time study, demonstrating the analytical model’s higher speed against EM simulations by being up to, e.g., approximately $55,000$ times faster for a $4\times4$ HELIOS with  $1^\circ$ angular resolution, thus enabling use cases as showcased throughout \Cref{Urban Case Study} as follows. A thorough comprehension of the urban setting is gained where we anticipate various communication channel models based on two unique reflector deployments: narrow and broad beam HELIOS. The overall coverage area in the south street canyon with FSPL channel modeling and radar realization for initial assessment from \cite{Helios} is very little, nearly \num{1} \%. The model's optimization concerning change in geometry and reflector position has greatly increased coverage regions, reaching up to 79.3\% with the broad beam HELIOS and 30.1\% with the narrow beam HELIOS. The examination of IRS models indicates the increase in the coverage area by all the models, especially the third IRS model \Cref{Model 3} with 100\% coverage. Further interpreting with UMi channel models using radar realization, a significant increase in coverage area is observed for optimized geometry and reflector placement with almost 98.6\% with the broad beam HELIOS and 94.5\% with the narrow beam HELIOS.
\section{Outlook}
During the derivation and evaluation of our analytical model, we pointed out effects by self-shadowing and diffraction. Whereas we already addressed the former, one may still extend the current implementation from horizontal/vertical shadowing to also include diagonal shadows, cf. \Cref{Improved Model using Inter-module Relations}. Further, this thesis recommends future work on including the diffraction behavior, but also reflection contributions of diffracted incident waves. For this purpose, existing edge and wedge diffraction concepts need to be investigated. Similarly, analytical models for trihedral corner reflectors which consider contributions of reflections of self-reflections are also a starting point for this \cite{Diff1, Diff2}.
 
 
 








%In this in-depth examination spanning numerous sections, we dug into the complex environment of mmWave communication and its propagation properties, establishing the groundwork for our work. \Cref{Channel Modeling} addressed key channel modeling ideas such as Free-Space Path Loss, 3GPP UMa, and UMi models. The radar realization of FSPL, as outlined in \Cref{FSPL_dB_Radar}, laid the groundwork for our following thesis work. Our attention then switched to metasurfaces in \Cref{Metasurfaces}, where we focused on active metasurfaces, including Intelligent Reflecting Surfaces (IRS), in \Cref{IRS}, and passive reflectors, including the HELIOS reflector idea, in \Cref{Passive reflectors}. IRS appeared as a prospective progression for 6G communications, but passive reflectors demonstrated a scalable, non-reconfigurable option with immediate viability.\\
%\Cref{coordinate systems} looked into metasurfaces for mmWave communication, establishing the framework for further talks on current IRS models, \Cref{Model 1}, \Cref{Model 2}, and \Cref{Model 3}, and their comparisons as plotted in \Cref{fig: compare 3 models} with their similarities and differences highlighted in \Cref{TABLE: comparison of the 3 models}. In \Cref{Simulation and Analytical Modeling of HELIOS Reflectors}, we investigate the simulation and analytical modeling of our HELIOS reflectors, explaining the EM simulation methodology and providing insights into their behavior under different conditions, for e.g., variation of slope angles in \Cref{Variation of Slope Angle}, variation of size in \Cref{Variation of Element Size}, and impact of reflecting surface materials in \Cref{Impact of Material Conductivity}. Notably, the HELIOS concept shows a shift in reflected angles as seen in \Cref{HELIOS 2times alpha}, and \Cref{HELIOS 2times beta}, further indicating the impact of increase in reflector size is directly proportional, so does the gain with a narrow beamwidth.\\
%\Cref{Analytical Modeling of HELIOS Modules} focuses on the analytical modeling of our HELIOS reflector module, which bridges the gap between simulations and theoretical structures. Equation for bistatic Radar Cross-Section (RCS) for flat plate reflector in \Cref{RCS_flatplate} was critical in the creation of an extended equation for HELIOS reflector modules in \Cref{Eq:HELIOS_module}, resulting in useful assessments that demonstrated an excellent agreement between simulation and analytical models. \Cref{Table:Comparison flat plate and HELIOS} provides the detailed assessment of the changes made to achieve the analytical equation for our HELIOS reflector module. The established formula provided in \Cref{Eq:RCS_SUMMATION} enables for superposition of the HELIOS module's separate RCS patterns, resulting in a formula for the whole HELIOS reflector array, as seen in \Cref{Independent Composition of Overall Reflection Characteristics}.\\
%\Cref{Improved Model using Inter-module Relations} addresses additional enhancements, with an emphasis on self-shadowing by reflector modules. The development of paired equations along rows and columns, as shown in \Cref{Table:Shadow region along the row}, and \Cref{Table:Shadow region along the column}, respectively, results in better performance, verifying our modified analytical model. Notable case studies includes the study to prove that there is no overprediction by our analytical model, the best choice of beamwidth variables, and scalability issues for bigger reflecting surfaces, which gives the complete insights.\\
%The analytical model's capacity to replicate simulation findings and adapt to complicated settings highlight its potential effect on the design and implementation of efficient and scalable communication networks.\\
%\Cref{Computing Time Analysis} presents a complete computation time study, demonstrating the analytical model's better speed and efficiency using multiple bar graphs, as seen in \Cref{fig:Computing_time_cores}, \Cref{fig:Computing_time_size}, and \Cref{fig:Computing_time_resolution}, where the discussion takes place for HELIOS reflector models between EM simulations with 4 core and 128 core and our analytical model. Moving on to \Cref{Urban Case Study}, a thorough comprehension of the urban setting is gained, as illustrated by \Cref{fig:Scenario1} and \Cref{Table:Urban case study}. We anticipated communication channel models in \Cref{Baseline Performance of FSPL Model} based on two unique reflector deployments, narrow and broad beams HELIOS, with a particular focus on FSPL and its radar realization in \Cref{fig:FSPL_Radar}. The visual depiction of RCS comparison of narrow and broad beams HELIOS of gain in \si{\decibel} and power received in \si{\decibel}m can be found in \Cref{fig:urbanscenario_originalvalues} for the traditional slope angle implementation in \cite{Helios}. Thus indicating very minor values in coverage area nearly \num{1} \%, notably in the south street canyon region. The model adjustment in \Cref{Optimizing the HELIOS Geometry for Better Performance}, and \Cref{Optimizing Connectivity with Different Reflector Positions} with respect to change in geometry and reflector position has greatly increased coverage regions, reaching up to 79.3\% with the broad beam and 30.1\% with the narrow beam.\\
%The examination of IRS models from \Cref{coordinate systems} in \Cref{Analyzing Urban Scenario for Different IRS Models} indicates the increase in coverage area by all the model, especially the IRS model 3 with 100\% coverage. \Cref{fig:Perfect_result_plot} shows a full comparison of all models, including HELIOS and IRS, using an ECDF plot. Further interpreting the urban scenario with different channel models, particularly 3GPP UMi using radar realization with \Cref{Umi_delta}, \Cref{Interpreting Different Channel Models} looked at the performance of our HELIOS and IRS models via RCS heatmap for power received, concluding with an ECDF plot summarizing their behavior in \Cref{fig:Perfect_result_plot_UMi}. The key observation was the significant increase in coverage area using the UMi channel model compared to FSPL, as seen in both the ECDF graphs.\\
%In conclusion, this thesis used a multidimensional approach, from the fundamentals of mmWave communication to the creation and validation of our analytical model for HELIOS reflectors. The findings highlight the effectiveness of passive reflectors, particularly HELIOS reflector modules, in improving communication networks. As we close this thesis, we not only reflect on our successes, but also realize the opportunities for future study and developments in the rapidly changing field of wireless communications.
%\section{Outlook}
%In investigating RCS gain behavior within the framework of our model, the thesis successfully replicated simulation results for a HELIOS reflector module, demonstrating the model's accuracy in capturing peak gains. However, when the model is extended to include an array of reflectors, problems arose, particularly when some phenomena, like as diffraction, were not properly accounted for, as seen in \Cref{fig:heatmaparraz}. The observed discrepancies reflect the array configuration's complexity and the need for more refining to capture nuanced behaviors.\\
%Another crucial issue that has yet to be completely addressed is the treatment of more complex self-shadowing effects. While the thesis addressed shadowing along rows and columns, see \Cref{Improved Model using Inter-module Relations}, the research did not account for complicated shadowing patterns along diagonals inside the array. This undiscovered dimension adds another degree of complexity to the model, since diagonal self-shadowing effects might have a major impact on the overall performance of the reflective array.\\
%This thesis provides significant insights into both unexplained diffraction phenomena and undiscovered diagonal self-shadowing subtleties. Addressing these issues would contribute to enhancing the model's accuracy and extending its application to more complicated settings. These challenges give intriguing potential for future research, helping to shape the continued evolution of passive reflectors and their role in sophisticated wireless communication networks.
